%CONTEUDO DO ARQUIVO TESE.TEX, NOSSO ARQUIVO PRINCIPAL
\documentclass[fleqn, a4paper, 12pt]{report}
\setlength{\mathindent}{0pt}

\input{custom/custom}

% %\listfiles
 \onehalfspacing %é pra ser o comando do espaçamento 1.5
 \setlength{\parindent}{1.5cm}
 \setlength{\hoffset}{-0.5cm}
 \setlength{\voffset}{-1cm}
 \setlength{\textwidth}{16cm}
 \setlength{\textheight}{23.5cm}
 

\begin{document}

\onehalfspacing
 \newgeometry{
 a4paper,
 top=3cm,
 bottom=1cm
 }
\pagenumbering{gobble}

\begin{titlepage}
  \begin{figure}[htb]			
	  \centering
	  \includegraphics[width=0.15\textwidth]{figuras/ifce.png}
  \end{figure}
  \begin{center}
    \textbf{INSTITUTO FEDERAL DE EDUCAÇÃO, CIÊNCIA E TECNOLOGIA DO CEARÁ}\\
    \textbf{IFCE \textit{CAMPUS} CEDRO}\\
    \textbf{LICENCIATURA EM MATEMÁTICA}\\

  \end{center}

 
\centering
	\begin{center}
	
		\vfill
		\textbf{NOME COMPLETO \\} 
\vfill
		\textbf{TÍTULO DO TRABALHO}\\
		
		
		
		
	\end{center}
	 \vfill
		
			\textbf{CEDRO - CE \\ 2021}
\end{titlepage}

\begin{titlepage}

\begin{center}
\textbf{LEVI CORDEIRO CARVALHO\\}
\vspace{7,5cm}

		\textbf{EXPLICAÇÕES PARA REDES NEURAIS BASEADAS EM RACIOCÍNIO ABDUTIVO}\\

\vspace{2.5cm}

\vspace{2cm}
\begin{flushright}
	\begin{minipage}[l]{8cm}
		Trabalho de Conclusão de Curso apresentado ao Curso de Bacharelado em Ciência da Computação do Instituto Federal de Educação, Ciência e Tecnologia do Ceará -IFCE - {\it{Campus}} Maracanaú como requisito parcial para obtenção do Título de Bacharel em Ciência da Computação.\\
		Orientador: Prof. Dr. Thiago Alves Rocha.
	\end{minipage}
\end{flushright}

\end{center}

\vfill
\centering
		
			\textbf{MARACANAÚ - CE \\2021}

\end{titlepage}


%Esta é a ficha catalográfica, Não sei como você e seu orientador optaram para colocá-la, a minha está como imagem, se for fazer o mesmo retire os porcentos. 

%\newpage
%	\begin{figure}[b]		
%\vspace{12.0cm}

		%\includegraphics[scale=0.70]{figures/fichaficha}
	%\end{figure}

%\includepdf[pages={1}]{fichacatalografica.pdf} %fornecida pela biblioteca
%\includepdf[pages={1}]{folhadeaprovacao.pdf} %gerado pelo tex folhaaprovacao.tex assinado e escaneado
\include{folhaaprovacao}

\include{dedicatoria}
\include{epigrafo}
\chapter*{}
\vspace{-4cm}
\begin{center}
 \textbf{AGRADECIMENTOS}
\end{center}

\vspace{0.5cm}
Graças à vida, que me deu tanto...

Ao Instituto Federal de Educação, Ciência e Tecnólogia do Ceará - \textit{Campus} Cedro, todos os servidores, professores e alunos.

Não esqueça de agradecer às instituições que lhe forneceram algum tipo de financiamento ao longo da graduação!!!




\include{resumo}

\listofilustr
\listoffigs
\listofcods
\newpage
\renewcommand{\listtablename}{\hfill \bf LISTA DE TABELAS\hfill}
%\listoftables
\newpage
%\renewcommand
\tableofcontents

 \newgeometry{
 a4paper,
 top=3cm,
 bottom=3cm
 }

 \pagenumbering{arabic}
% \clearpage
\setcounter{page}{12}
  \chapter{INTRODUÇÃO}
\thispagestyle{empty}

Introdução ao trabalho explicando os objetivos e estrutura do texto.
  \chapter{MOTIVAÇÃO INICIAL E JUSTIFICATIVA FUNDAMENTADA DO ESTUDO}
\label{cap1}
\thispagestyle{empty}

Apresente o que motivou o estudo e a justificativa da relevância e necessidade do estudo. O Quadro \ref{tabela-livro} é uma exemplo de quadro.

\BQUAD{LIVROS ANALISADOS}
  \begin{tabular}{|p{3cm}||c||p{4cm}|}
      \hline
      \textbf{Referência para citar no texto}    & \textbf {Título do livro} & \textbf {Autor/Autores} \\
  \hline
  \hline
  \label{livro 1}Livro 1 & Matemática Completa&Bonjorno, Giovanni Jr e Paulo Câmara \\ 
  \hline
 \label{livro 2}Livro 2 & Matemática: Contexto e Aplicações & Luiz Roberto Dante\\ 
 \hline
 \label{livro 3}Livro 3 & Matemática & Emanuel Paiva \\ 
 \hline
  \label{livro 4}Livro 4 & Matemática: Ciência e Aplicações  &Gelson Iezzi, Osvaldo Dulce, David Degenszajn, Roberto Périgo e Nilse de Almeida \\ 
  \hline
  \label{livro 5}Livro 5 & Matemática para compreender o mundo & Kátia Stocco Smole e Maria Ignez Diniz \\ 
 
 \hline
  \label{livro 6}Livro 6 & Fundamentos de Matemática Elementar & Gelson Iezzi e Carlos Marukami \\ 
 \hline

 \end{tabular}
\EQUAD{Fonte: Elaborado pela autor(a).}{tabela-livro}
\vspace{0.5cm}


Um exemplo de lista de itens.

\begin{itemize}
 \item [Livro 1 -\nocite{matcompleta}]  Neste livro, ... Ao final das seções percebemos razoável variação de exercícios resolvidos e propostos.
\vspace{0.5cm}
	 
\item [Livro 2 -\nocite{dante2010matematica}] No segundo livro, ... 
\vspace{0.5cm}

\item [Livro 3 -\nocite{paiva2010matematica}]   Nesse exemplar ... 
\vspace{0.5cm}

\item [Livro 4 - ]   \citeonline{iezzi2016}, ...
\vspace{0.5cm}

\item [Livro 5 -\nocite{smole2016matematica}] As autoras abordam ....
\vspace{0.5cm}

\item [Livro 6 -]  Neste livro, ...
\vspace{0.5cm}
 \end{itemize}

 
  \include{cap2} 
  \include{cap3} 
 % \include{cap4} % Ativar caso queira o cap 4
  \include{conclusao}
  \appendix 
  \renewcommand{\thechapter}{ANEXO A}

  \include{ApendixA} 
  \renewcommand{\thechapter}{APÊNDICE B}
 % \include{ApendixB} 
  \renewcommand{\thechapter}{APÊNDICE C}
 % \include{ApendixC} 
  %\printindex 
  %\include{refer}
%  \printbibliography
  %\include{refs}  
  \bibliography{refs}
  \addcontentsline{toc}{chapter}{REFERÊNCIAS}

\end{document}  
